% !TeX root = ../main.tex

\chapter{Introdução \label{chap:INTRO}}

% apresenta o capítulo como faz nos outros

A visão é um sentido importante para os primatas que têm inspirado toda uma área de pesquisa denominada visão computacional, cujas origens e evolução, intimamente relacionadas com a história da computação, tem recentemente motivado o desenvolvimento de tecnologias com uma vasta gama de aplicações \cite{Costa:2009}. A visão é um processo complexo, pois envolve a análise de várias informações visuais, como a cor, profundidade, movimento, forma e textura dos objetos, com o propósito de locomoção, reconhecimento, classificação e manipulação dos objetos \cite{Ullman:1996}.

Dentre as informações visuais, a forma desempenha um papel importante no sistema de percepção visual humano pela riqueza de informação que esta propicia. A forma advém da imagem resultante da projeção dos objetos existentes no espaço tridimensional em estruturas bidimensionais, como a retina do olho humano ou um sensor de \acf{CCD}  de uma câmera \cite{Costa:2009}.

Em várias aplicações \cite{Zhang201661,Zhao20153203}, informações sobre a forma são suficientes para caracterizar os objetos existentes em uma cena, independente dos demais atributos da imagem. Na medicina, descritores de forma foram aplicados no diagnóstico de doenças da coluna vertebral \cite{Lee:2009}, na avaliação clínica de tumores de mama a partir de imagens de ultrassom \cite{Yang:2009} e na identificação automática de pílulas \cite{Ushizima:2015}.  Descritores de forma também foram aplicados, em microscopia, na análise de materiais \cite{Zhang201661} e em recuperação de informação, na busca de marcas registradas por similaridade de conteúdo \cite{MohdAnuar2013105,Qi20102017}.

Uma área de particular interesse em aplicações de visão computacional é o processo de análise e reconhecimento de formas, que envolve a descrição computacional e a avaliação da similaridade das formas. Esse processo consiste em detectar e representar atributos visuais das imagens que sejam relevantes para melhoria do desempenho dos algoritmos que realizam as tarefas de classificação e reconhecimento dos objetos \cite{Escolano:2009}. Quanto à avaliação de similaridade, procura-se por medidas que sejam semelhantes ao sistema de percepção da visão humana, ainda que não esteja totalmente claro como tal sistema opera \cite{4815272}. 

Há dois tipos de métodos de descrição de formas: os baseados em região e os baseados em contorno \cite{Zhang:2004}. Os baseados em região extraem características de toda a área interior à forma. Alguns métodos desta natureza são os momentos de Zernike \cite{Kim:2000} e os momentos de Legendre \cite{Yang:2006} que, apesar de apresentarem excelente desempenho, são inadequados para reconhecimento de objetos na presença de oclusões. Utilizando as propriedades da transformada de Fourier, o descritor genérico de Fourier \cite{Zhang:2002} permite a análise multiescala de formas. Já os métodos baseados em contorno exploram a informação contida na fronteira entre o interior da forma e o fundo da imagem. Embora mais complexos, pois requerem implementações mais sofisticadas, esses métodos são mais adequados para o reconhecimento de objetos com oclusões. Nesta categoria, encontramos a codificação em cadeia \cite[p~337]{Costa:2009}, que consiste em segmentos de linhas mapeados em uma grade fixa com um número finito de possíveis orientações, as aproximações poligonais \cite[p~340--351]{Costa:2009} e os métodos de esqueletização \cite[p~394--400]{Costa:2009}. 

Na abordagem por aproximações poligonais, a forma é decomposta em segmentos de linha. Os vértices dos polígonos são utilizados como primitivas de onde algumas características são extraídas. A transformação do eixo mediano, ou esqueletização, foi introduzido por \citeonline{blum:1967}. Esta consiste em reduzir regiões a curvas que seguem a forma global de um objeto. \citeonline{Sebastian:2004} utilizaram este descritor no reconhecimento de formas. \citeonline{Milios:2000} introduziram a representação as formas como uma coleção de segmentos entre dois pontos de inflexão consecutivos. Os segmentos obtidos são considerados em diferentes níveis de resolução. 

% Os métodos acima, que aproximam as formas como polígonos e as representam a partir de conjuntos de segmentos de linha, funcionam bem para objetos construídos pelo homem, mas não são adequados para representar objetos naturais \cite{Zhang:2004}.

A descrição multiescala de formas \cite{LiKuangLiuEtAl2016,ShuPanWu2015, HuangHanHe2014, Costa:2009} é uma abordagem promissora em reconhecimento de padrões em imagens \cite{Direkoglu:2011}. Na descrição multiescala os atributos das formas são representados em vários níveis de detalhes, variando de escalas de baixa resolução, aonde os detalhes que diferenciam as formas de uma mesma classe não são levados em consideração, até escalas de alta resolução aonde esses detalhes são preservados \cite{Ullman:1996}. Desta forma, a combinação desses atributos, de baixa e alta resolução, aumentam o poder de discriminação do descritor, melhorando o desempenho na tarefa de classificação \cite{Direkoglu:2011}.

\citeonline{YangWangYuanEtAl2016} também apresentam uma proposta robusta de descritor multiescala para a recuperação de formas pelo conteúdo, com propriedades de invariância à rotação, escala, variação intraclasse, deformação,  oclusão parcial e ruídos. Esta proposta utiliza três assinaturas invariantes de contorno para capturar características locais e globais das formas em múltiplas escalas, sendo a avaliação da similaridade entre as assinaturas feita através de programação dinâmica.


\citeonline{Mokhtarian:1986} propuseram o descritor \ac{CSS}, que é baseado na representação multiescala da curvatura das formas. O \ac{CSS} é invariante às transformações afins, mas é sensível a oclusão e convexidade das formas. A \ac{TAR} \cite{Alajlan20117} é outro tipo de representação multiescala baseado nas áreas de triângulos formados pelos pontos das borda, calculados para diferentes escalas. 

Outras técnicas de representação do contorno das formas são o \ac{d-TAS} \cite{4202050}, \textit{b-splines} \cite{1168520} e as funções peso, ou \textit{height functions} \cite{Wang2012134}. A função peso de um ponto amostral do contorno é definida como sendo pelas distâncias de todos os outros pontos amostrais a sua linha tangente. A função peso obtida é então suavizada para representar e reconhecer objetos bidimensionais

Um método clássico de encontrar correspondência entre conjuntos de pontos, que utiliza a distância euclidiana como métrica para construção da matriz de distâncias, é o \ac{SC} \cite{Belongie:2002}. Já \citeonline{1467513} propuseram o \ac{IDSC} substituindo a distância euclidiana do \ac{SC} pela distância do produto escalar. Embora ambos os métodos tenham a habilidade de extrair características bastante discriminativas das formas, lidando também com o problema da inexatidão de correspondência, estes são sensíveis a diferentes deformações e poses de uma mesma forma. Ademais, por utilizarem programação dinâmica, esses métodos apresentam um custo computacional elevado na avaliação da similaridade entre as formas \cite{FreitasS.TorresMiranda2016}.

Variantes do \ac{SC} e do \ac{IDSC} são encontrados em \citeonline{Nanni20122254}. O método desses autores transforma os descritores obtidos a partir da \ac{IDSC}, \ac{SC} e funções de peso em um descritor matricial por quantização local de fase. Os descritores matriciais das formas são então comparados através do divergente de Jeffrey \cite{Ullah1996}. \citeonline{Hu20123348} propuseram um descritor baseado em contorno para o reconhecimento de formas de mão, o qual foi nomeado de distância coerente do contexto da forma, do inglês \ac{CDSC}. O método \ac{CDSC} é robusto a diferentes poses da mão e pode ser utilizado tanto para o reconhecimento das formas como das palmas das mãos.  

A representação das formas por suas partes constituintes tem desempenhado um papel relevante no reconhecimento de objetos \cite{Ullman:1996}. Tal abordagem resulta em uma representação robusta das formas, pois, além de representar suas partes constituintes, representa as relações espaciais existentes entre as partes.  

Em \citeonline{Kenji:1992}, as formas são decompostas em diferentes retângulos, sendo a localização e as dimensões dos retângulos selecionadas por programação dinâmica. \citeonline{Berretti:2000} propuseram o uso dos pontos de cruzamento de zero da curvatura de um contorno suavizado para obtenção de partes cujos autores denominaram de fichas ou \textit{tokens}. As orientações e os pontos de máxima curvatura das partes obtidas são levadas em consideração para representar e comparar as formas. Esse método não é invariante a rotação devido a orientação das fichas ou \textit{tokens} \cite{DiRuberto:2009}. 

Usando programação dinâmica, \citeonline{Latecki:2007} propuseram um método para comparação parcial de formas aonde tangentes locais ao contorno são utilizadas na descrição de formas. \citeonline{Cui:2009} propuseram o uso da integral da curvatura absoluta como descritor de forma. Para comparar as partes das curvas eles utilizaram a correlação cruzada, sendo o método invariante a rotação, escala e translação.  

\section{Motivação e objetivos \label{sec:motiv_obj}}
%1.1 MOTIVAÇÃO E OBJETIVOS (Na motivação vc enxerta o que já tem na INTRO inclui o texto que já está aqui  como motivação e pincela um pouco sobre o estado da arte que sustenta o trabalho. Como é um doutorado deve ter papers de 2015 e 2016.

Descritores de forma \cite{Belongie:2002, 1467513, Nanni20122254, Hu20123348, Latecki:2007, Wang2012134}, em sua maioria, são utilizados em sistemas de busca e recuperação de imagens pelo conteúdo (\ac{CBIR}). Sistemas dessa natureza realizam buscas em bases multimídia utilizando o conteúdo visual das imagens para recuperar aquelas que sejam do interesse do usuário mediante um padrão de consulta por ele especificado \cite{Feng:2003}. 

Em termos de aplicações de descritores de formas em \ac{CBIR}, destacamos, em particular, neste trabalho, a identificação de espécies vegetais através das folhas \cite{deSouza2016,
Fotopoulou:2013,Zhao20153203, Nam2008245, Wang:2000}. A abordagem tradicional para a classificação de espécies vegetais é a taxonomia, ou seja, o treinamento de especialistas, denominados taxonomistas, para a identificação das espécies através das características observadas nas plantas \cite{Cope20127562}. A deficiência dessa abordagem é o aspecto subjetivo do taxonomista \cite{JVS:JVS1441} e o pequeno número de taxonomistas disponíveis \cite{Cope20127562}. Ademais, os taxonomistas são especialistas em identificar um pequeno grupo de espécies, sendo assim impossível aos mesmos classificarem diversas espécies \cite{Cope20127562}.

Historicamente, as amostras de espécies são armazenadas em arquivos físicos, denominados de herbários \cite{Cope20127562}, para posterior classificação, tornando o acesso às amostras difícil e lento. Com a criação das bases de dados digitais, a informação tornou-se facilmente acessível aos pesquisadores em qualquer parte do mundo, o que permite o desenvolvimento de métodos computacionais automatizados para classificação das espécies a partir de imagens. Desta forma, sistemas dessa natureza permitiriam que pessoas com um conhecimento limitado de botânica desempenhem satisfatoriamente em campo tarefas de identificação de espécies que requereriam o trabalho de um taxonomista \cite{Cope20127562}.  


%Pesquisas recentes \cite{Principe:2014,Zang:2014} tem explorado conceitos da teoria da informação, como entropia e medidas de divergência, em análise de formas \cite{Escolano:2009}, pois essas medidas se relacionam diretamente com as informações contidas nos sinais \cite{Principe:2011}. \citeonline{Page:2003} utilizam a entropia de Shannon da curvatura em análise de formas para medir a complexidade de formas planas. \citeonline{1716783} aplicam conceitos da teoria da informação, como divergência e entropia, para medir similaridade entre formas e imagens \cite{Principe:2014,Zang:2014}.


%Este trabalho explora técnicas para representação computacional de características das formas para fins de reconhecimento de padrões, em particular os algoritmos multiescala de representação do contorno das formas.  Enfatizamos também as técnicas de representação do contorno por assinaturas, uma vez que descritores multiescala podem ser obtidos a partir dessas assinaturas. Este trabalho introduz uma metodologia de otimização para ajuste de  parâmetros do descritor energia de dobramento multiescala (NMBE) \cite{Costa:1997} no contexto da aplicação em análise de folhas para reconhecimento de espécies vegetais. A maior contribuição deste trabalho é prover um método para ajuste de parâmetros de descritores multiescala de forma através de otimização evolutiva computacional.
 

%Divergentes são funcionais que medem a distância existente entre modelos estatísticos que são utilizadas em processamento de sinais e reconhecimento de padrões \cite{Basseville1989349}.   

%Como medidas de similaridade avaliamos métricas clássicas {\color{red} CITAR TRABALHOS IMPORTANTES E REVISTAS ISI}, como a norma $L_2$, bem como medidas de divergência estocástica{\color{red} CITAR FONTES ATUAIS SE TIVER}  {\color{red}}. Essas últimas medem o grau de similaridade existente entre distribuições de probabilidade. Embora sejam empregadas em análise estatística de dados, processamento de sinais e teoria da informação, as medidas de divergência têm sido pouco exploradas na avaliação de similaridade entre formas, em particular no contexto de recuperação de imagens pelo conteúdo.  


%Uma área de pesquisa particularmente relevante em visão computacional é a que trata do desenvolvimento de algoritmos capazes de avaliar o grau de similaridade existente entre as formas de maneira análoga àquela realizada pelos humanos. Tal empreendimento envolve representar, ou descrever, computacionalmente o conteúdo dessas imagens e obter medidas que possibilitem avaliar a similaridade entre as mesmas com base em suas representações. Esses dois aspectos são o foco desse trabalho de pesquisa.

Um aspecto importante em análise de formas é a adequação do descritor ao problema de reconhecimento de padrões de interesse. Tal tarefa envolve ajuste de parâmetros, assim como a investigação da estrutura que o descritor impõe aos dados. Desta forma, busca-se ajustar a descrição ao domínio de conhecimento do problema, ou seja, colocá-la em conformidade com a percepção dos especialistas da área. 

Muitas vezes, no projeto de descritores de forma, atribui-se valores de parâmetros que independem da base de dados representada. Em outras situações, esse ajuste é obtido através de um processo empírico ou mesmo por busca exaustiva \cite{mokhtarian1998robust, Ling:2007:SCU:1191552.1191806, Wang2012134}.

 \citeonline{Paula:2013} utilizou um esquema de otimização por busca exaustiva e reportou resultados satisfatórios no ajuste de parâmetros de um descritor a uma base de imagens de formas.  \citeonline{4815272} propuseram um método para ajuste dos parâmetros de descritores de forma por aprendizagem supervisionada. Com isso a similaridade entre padrões de formas pré-estabelecidos foi utilizada em um processo de aprendizagem sensível ao contexto e que busca os parâmetros mais adequados ao cálculo da matriz de similaridade de um conjunto de formas \cite{4815272}.

O ajuste de parâmetros dos descritores pode ser considerado um problema de otimização e \ac{CI} \cite{Andries:2007}, sendo esta uma subárea da inteligência artificial. Estudos nessa área envolvem o desenvolvimento de algoritmos bio-inspirados, que apresentam comportamento inteligente, para resolução de problemas complexos. Em otimização, os algoritmos de \ac{EC} imitam a evolução natural dos processos de populações. Desta forma, cada solução candidata é análoga a um indivíduo em uma população e sua qualidade, à aptidão desse indivíduo \cite{Eiben:2015}.

Este trabalho tem como objetivo geral introduzir uma metodologia para o ajuste de parâmetros dos descritores multiescala de formas \cite{Costa:2009}, através de otimização evolutiva \cite{Andries:2007}, para adequação dos mesmos a um problema de visão computacional que seja de interesse. Assim, investigamos a aplicabilidade do referido método ao problema de descrição multiescala da forma das folhas para a identificação das espécies de plantas. 

%O método proposto busca ajustar os parâmetros dos descritores para propiciar a melhor configuração possível de arranjo das formas no espaço de representação multidimensional do descritor. 

%Embora sejam importantes, as métricas de avaliação utilizadas em classificação, como a acurácia, a precisão e a revocação  \cite{Ting2010}, não revelam aspectos importantes que são do interesse desse trabalho, como o grau de separação entre classes e coesão intraclasse que o descritor propicia \cite{Meta:2009}. A propriedade de separação entre classes diz respeito à habilidade do descritor em diferenciar formas que não sejam similares e, portanto, pertencentes a classes distintas. Já a coesão intraclasse indica o potencial do descritor em representar formas similares como pertencentes a uma mesma classe.  Medidas de qualidade de agrupamentos \cite{Meta:2009} avaliam tais propriedades quantitativamente, enquanto o uso de técnicas de projeção de dados \cite{Amorim201535} permitem avaliá-las qualitativamente. Através dessas últimas técnicas, é possível visualmente inferir a estrutura que o descritor impõe aos dados, quando os representa num espaço vetorial multidimensional. 

Quanto aos objetivos específicos destacamos:

\begin{itemize}
\item Aplicar a metodologia de otimização ao descritor de formas \acf{NMBE} \cite{Costa:2009} e \acf{IDSC} \cite{1467513};

\item Propor uma função objetivo, a ser empregada nos algoritmos de otimização, para busca de soluções ótimas para os conjuntos de parâmetros multiescala dos descritores de formas;

\item Avaliar qualitativamente e quantitativamente a capacidade discriminativa dos descritores multiescala otimizados e não otimizados;

\item Empregar técnicas de visualização de dados e medidas de avaliação da qualidade de agrupamentos para quantificar a capacidade discriminativa dos descritores estudados; 

\item Realizar experimentos de classificação e recuperação de formas pelo conteúdo, em bases públicas de imagens de uso geral e de folhas de plantas, para quantificar a qualidade dos descritores;

\end{itemize}



%%%%%%%%%%%%%%%%%%%%%%%%%%%%%%%%%%%%%%%%%%%%%%%%%%%%%%%%%%%%%%%%%%%%%%%%%%%%%%%%%%%%%
% Da qualificação
%%%%%%%%%%%%%%%%%%%%%%%%%%%%%%%%%%%%%%%%%%%%%%%%%%%%%%%%%%%%%%%%%%%%%%%%%%%%%%%%%%%%%
\section {Produção Científica \label{sec:prod_cientif}}
\begin{itemize}

\item SOUZA, M.M.S.; MEDEIROS, F.N.S.; RAMALHO, G.L.B.; DE PAULA JR., I.C.; Oliveira, I.N. Evolutionary optimization of a multiscale descriptor for leaf
shape analysis. Expert Systems with Applications, p. –, 2016. ISSN 0957-4174. Disponível
em: <http://www.sciencedirect.com/science/article/pii/S095741741630361X>.

\item USHIZIMA, D.; CARNEIRO, A.; SOUZA, M.; MEDEIROS, F. Investigating pill recognition methods for a new national library of
medicine image dataset. In: Advances in Visual Computing. Springer Science Business Media, 2015. p 410-419. Disponível em: <http://dx.doi.org/10.1007/978-3-319-27863-6\_38>


\item NOGUEIRA, O.C.; SOUZA, M.M.S. ; MEDEIROS, F.N.S. ; OLIVEIRA, I.N.S. Dimensão Fractal em Recuperação de Imagens Baseadas em Conteúdo (CBIR). XIII Escola Regional de Informática da SBC, v. 1, p. 1-9, 2013.

\end{itemize}
\section{Organização da Tese \label{sec:org}}

A organização dos capítulos desta tese segue a seguinte estrutura:

\noindent \textbf{Capítulo \ref{chap:FUNDA}}: contém a descrição de conceitos teóricos e ferramentas utilizadas no desenvolvimento, avaliação e testes dos descritores, assim como no desenvolvimento da metodologia proposta para análise de formas.

\noindent \textbf{Capítulo \ref{chap:MatMet}}:  introduz a metodologia proposta para ajuste automático de parâmetros multiescala de formas por algoritmos de otimização evolucionária, bem como para a avaliação do descritor resultante desse processo de otimização. Apresentamos ainda a função objetivo a ser minimizada na busca de soluções ótimas para o problema.

\noindent \textbf{Capítulo \ref{chap:resultados}}: expõe e discute os resultados obtidos pela aplicação do método proposto ao problema de classificação de espécies vegetais com os descritores de forma \ac{NMBE} e \ac{IDSC}.

\noindent \textbf{Capítulo \ref{chap:ch5}}: traz as conclusões do trabalho e as sugestões de pesquisas futuras. 
