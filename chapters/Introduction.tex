% !TeX root = ../main.tex

\chapter{Introdução \label{chap:INTRO}}

% apresenta o capítulo como faz nos outros

{\color{red} TENTEI AJEITAR AS REFS ERRADAS DO CESAR JR AND COSTA , AJEITEI NO TESE.BIB, MAS NO TEXTO CONTINUAVA ERRADO, FIZ DO MEU JEITO}

Como principal sentido humano, a visão tem inspirado toda uma área de pesquisa denominada visão computacional, cuja origem e evolução, intimamente relacionada com a história da computação, tem recentemente motivado o desenvolvimento de tecnologias com uma vasta gama de aplicações \cite{Costa:2009}. A visão é um processo complexo, pois envolve a análise de várias informações visuais, como a cor, profundidade, movimento, forma e textura dos objetos, com o propósito de locomoção, reconhecimento, classificação e manipulação dos objetos \cite{Ullman:1996}.

Dentre as informações visuais, a forma desempenha um papel importante no sistema de percepção visual humano pela riqueza de informação que esta propicia. A forma advém da imagem resultante da projeção dos objetos existentes no espaço tridimensional em estruturas bidimensionais, como a retina do olho humano ou um sensor de \emph{CCD} de uma câmera.

Uma área de particular interesse em aplicações de visão computacional é o processo de análise e reconhecimento de formas, que envolve representá-las computacionalmente para avaliar, com base nessa representação, o grau de similaridade entre as mesmas. Esse processo consiste em detectar e representar pistas visuais das imagens suficientemente informativas para melhorar o desempenho dos algoritmos que realizam as tarefas de classificar, comparar ou reconhecer os objetos.\cite{Escolano:2009}.

Outro aspecto importante no reconhecimento de formas são as medidas de avaliação de similaridade. Destas procuram-se aquelas que sejam semelhantes ao sistema de percepção da visão humana, ainda que não esteja totalmente claro como tal sistema opera \cite{4815272}. 

Este trabalho explora técnicas para representação computacional de características das formas para fins de reconhecimento de padrões. O atual estado da arte em representação de formas é apresentado nesta introdução na Seção \ref{sec:est_art}. A Seção \ref{sec:motiv_obj} comenta quais aspectos motivaram o desenvolvimento deste trabalho de pesquisa. Já a Seção \ref{sec:contrib} apresenta as contribuições do trabalho, e a Seção \ref{sec:prod_cientif} a produção científica decorrente do mesmo. Finalizando esta introdução, temos na Seção \ref{sec:org} a estruturação da tese.

\section {Estado da arte \label{sec:est_art}}
   
{\color{red} A DEFINIR O LUGAR ERA PARTE DA Motivação}

Há dois tipos de métodos de descrição de formas: os baseados em região e os baseados em contorno \cite{Zhang:2004}. Os baseados em região extraem características de toda a área interior à forma. Alguns métodos desta natureza são os momentos de Zernike \cite{Kim:2000} e os momentos de Legendre \cite{Yang:2006} que, apesar de apresentarem excelente desempenho, são inadequados para reconhecimento de objetos na presença de oclusões. Utilizando as propriedades da transformada de Fourier, o descritor genérico de Fourier \cite{Zhang:2002} permite a análise multiescala de formas.

Em contraste, os métodos baseados em contorno exploram a informação contida na região da fronteira entre a forma e o fundo da imagem. Embora mais complexos, pois requerem implementações mais sofisticadas, esses métodos são mais adequados para o reconhecimento de objetos com oclusões. Nesta categoria encontramos a codificação em cadeia \cite[p~337]{Costa:2009}, que consiste em segmentos de linhas mapeados em uma grade fixa com um número finito de possíveis orientações, as aproximações poligonais \cite[p~340--351]{Costa:2009} e os métodos de esqueletização \cite[p~394--400]{Costa:2009}. 

Na abordagem por aproximações poligonais, a forma é decomposta em segmentos de linha. Os vértices dos polígonos são utilizados como primitivas donde algumas características são extraídas. A transformação do eixo mediano, ou esqueletização, foi introduzido por \citeonline{blum:1967}. Esta consiste em reduzir regiões a curvas que seguem a forma global de um objeto. \citeonline{Sebastian:2004} utilizaram este descritor no reconhecimento de formas. \citeonline{Milios:2000} propuseram representar as formas como uma coleção de segmentos entre dois pontos de inflexão consecutivos. Os segmentos obtidos são considerados em diferentes níveis de resolução. 

Os métodos acima, que aproximam as formas como polígonos e as representam a partir de conjuntos de segmentos de linha, funcionam bem para objetos construídos pelo homem, mas não são adequados para representar objetos naturais \cite{Zhang:2004}.

A descrição multiescala de formas é uma abordagem promissora em reconhecimento de padrões em imagens \cite{Direkoglu:2011}. Na descrição multiescala os atributos das formas são representados em vários níveis de detalhes, variando de escalas de baixa resolução, aonde os detalhes que diferenciam as formas de uma mesma classe não são levados em consideração, até escalas de alta resolução aonde esses detalhes são preservados \cite{Ullman:1996}. Desta forma, a combinação desses atributos, de baixa e alta resolução, aumentam o poder de discriminação do descritor, melhorando o desempenho na tarefa de classificação \cite{Direkoglu:2011}.

\citeonline{Mokhtarian:1986} propuseram o descritor \textit{CSS}, que é baseado na representação multiescala da curvatura das formas. A representação \textit{CSS} é invariante às transformações afins, mas é sensível a oclusão e convexidade das formas. Representação pela área de triângulos (\textit{TAR}) \cite{Alajlan20117} é outro tipo de representação multiescala baseado nas áreas de triângulos formados pelos pontos das borda, calculados para diferentes escalas. 

Outras técnicas consistem em aproximar o contorno da forma através de uma função espaço-escala do ângulo de giro (\textit{d-TAS}) \cite{4202050}, \textit{b-splines} \cite{1168520} e as funções peso, ou \textit{height functions} \cite{Wang2012134}. A função peso de um ponto amostral do contorno é definida como sendo pelas distâncias de todos os outros pontos amostrais a sua linha tangente. A função peso obtida é então suavizada para representar e reconhecer objetos bi-dimensionais.

\textit{Shape context} (\textit{SC}) \cite{Belongie:2002} é um método clássico de encontrar correspondência entre conjuntos de pontos. Quando baseado na distância do produto escalar, \textit{SC} se torna \textit{inner distance shape context} (\textit{IDSC}) \cite{1467513}. Esses métodos tem a habilidade de extrair características bastante discriminativas para uma forma e lida com o problema da inexatidão de correspondência na comparação entre formas. Porém, estes são sensíveis a diferentes deformações e poses de uma mesma forma.

Diversas variantes do \textit{SC} e \textit{IDSC} são encontrados na literatura. Recentemente, \citeonline{Nanni20122254} apresentaram um novo método de representação de formas que transforma os descritores obtidos a partir da \textit{IDSC}, \textit{SC} e funções de peso em um descritor matricial através de quantização local de fase. Os descritores matriciais são então comparados através da distância de Jeffry. \citeonline{Hu20123348} propuseram um descritor baseado em contorno para o reconhecimento de formas de mão. Chamado pelos referidos autores de \textit{Coherent Distance Shape Context} (\textit{CDSC}), esse método baseia-se no \textit{SC} e no \textit{IDSC}. O \textit{CDSC} é robusto a diferentes poses da mão e pode ser utilizado tanto para o reconhecimento das formas como das palmas das mãos.  

As formas podem também ser modeladas através da representação por partes. Esta abordagem tem desempenhado um papel relevante no reconhecimento de objetos. Organizar a representação das formas em termos de suas partes constituintes permite separar a representação de cada uma das partes constituintes da representação das relações espaciais existentes entre as partes. Isso, por sua vez, resulta em uma representação mais robusta da forma. 

Em \cite{Kenji:1992} as formas são decompostas em diferentes retângulos. A localização dos retângulos e suas dimensões são selecionadas por programação dinâmica. \citeonline{Berretti:2000} propuseram o uso dos pontos de cruzamento de zero da curvatura de um contorno suavizado para obtenção de partes denominadas de \textit{tokens}. As orientações e os pontos de máxima curvatura das partes obtidas são levadas em consideração para representar e comparar as formas. Esse método não é invariante a rotação devido a orientação dos \textit{tokens} \cite{DiRuberto:2009}. Usando programação dinâmica, \citeonline{Latecki:2007} propuseram um método para comparação parcial de formas aonde tangentes locais ao contorno são utilizadas na descrição de formas. \citeonline{Cui:2009} propuseram o uso da integral da curvatura absoluta como descritor de forma. Para comparar as partes das curvas eles utilizaram cross correlação, sendo o método invariante a rotação, escala e translação.  

\section{Motivação e objetivos \label{sec:motiv_obj}}
%1.1 MOTIVAÇÃO E OBJETIVOS (Na motivação vc enxerta o que já tem na INTRO inclui o texto que já está aqui  como motivação e pincela um pouco sobre o estado da arte que sustenta o trabalho. Como é um doutorado deve ter papers de 2015 e 2016.
Em várias aplicações, informações sobre a forma são suficientes para caracterizar os objetos existentes em uma cena independente dos demais atributos da imagem \cite{Zhang201661,deSouza2016,Zhao20153203}. Na medicina, descritores de forma foram aplicados no diagnóstico de doenças da coluna vertebral \cite{Lee:2009}, na avaliação clínica de tumores de mama a partir de imagens de ultrassom \cite{Yang:2009} e na identificação automática de pílulas \cite{Ushizima:2015}. Em microscopia, descritores de forma foram aplicados na análise de materiais particularizado \cite{Zhang201661}. 

Grande parte dos descritores de forma \cite{Belongie:2002, 1467513, Nanni20122254, Hu20123348, Latecki:2007, Wang2012134} são voltados para a utilização em sistemas de busca de imagens pelo conteúdo ou, do inglês,\foreignlanguage{english}{\emph{Content-based image retrieval}} (\emph{CBIR}). Sistemas dessa natureza realizam buscas em bases multimídia utilizando o conteúdo visual das imagens para recuperar aquelas que sejam do interesse do usuário mediante um padrão de consulta por ele especificado \cite{Feng:2003}. Em termos de aplicações de descritores de formas em \emph{CBIR}, destacamos a identificação de espécies vegetais através das folhas \cite{deSouza2016,
Fotopoulou:2013,Zhao20153203, Nam2008245, Wang:2000} e a recuperação de informação de marcas registradas por similaridade de conteúdo \cite{MohdAnuar2013105,Qi20102017}.

Pesquisas recentes \cite{Principe:2014,Zang:2014} tem explorado conceitos da teoria da informação, como entropia e medidas de divergência, em análise de formas \cite{Escolano:2009}, pois essas medidas se relacionam diretamente com as informações contidas nos sinais \cite{Principe:2011}. \citeonline{Page:2003} utilizam a entropia de Shannon da curvatura em análise de formas para medir a complexidade de formas planas. \citeonline{1716783} aplicam conceitos da teoria da informação, como divergência e entropia, para medir similaridade entre formas e imagens \cite{Principe:2014,Zang:2014}.

Grande parte dos trabalhos avaliam a qualidade dos descritores de forma por métricas de classificação como acurácia, precisão e revocação  {\color{red} CITAR FONTE, POIS VC SEQUER DISSE AINDA O QUE É REVOCAÇÃO PRECISÃO ACURÁCIA, PONHA FONTES ENTÃO}. Entretanto, tais medidas não possibilitam inferir propriedades importantes como o grau de separação entre-classes e coesão intra-classe que o descritor propicia \cite{Meta:2009}. A propriedade de separação entre-classes diz respeito à habilidade que o descritor tem em diferenciar formas que não sejam similares e, portanto, pertencentes a classes distintas. A coesão intra-classe indica o potencial do descritor em representar formas similares como pertencentes a uma mesma classe.  

No entanto, pode-se inferir tais propriedades quantitativamente e qualitativamente através de medidas de qualidade de agrupamentos \cite{Meta:2009} e técnicas de projeção, respectivamente. Estas últimas possibilitam visualizar os dados multidimensionais em um espaço de dimensão reduzida a fim de inferir a estrutura que a representação impõe aos dados no espaço vetorial multidimensional.

Este trabalho explora métodos multiescala de descrição de forma para fins de reconhecimento de padrões, aonde conceitos da teoria da informação são utilizados para esta finalidade. Assim, propomos um novo descritor multiscala de formas baseado em entropia.  Investigamos também o uso das medidas de divergência na avaliação da similaridade entre formas. 
Neste trabalho exploramos duas técnicas, classicamente empregadas em visualização de dados, para investigar a qualidade dos descritores de formas: a análise das componentes principais (\emph{PCA}) e o mapa auto-organizável de Kohonen (\textit{SOM}) \cite{Kohonen:1982}.   

Este trabalho explora técnicas para representação computacional de características das formas para fins de reconhecimento de padrões, em particular os algoritmos multiescala de representação do contorno das formas.  Enfatizamos também as técnicas de representação do contorno por assinaturas, uma vez que descritores multiescala podem ser obtidos a partir dessas assinaturas. 

%Divergentes são funcionais que medem a distância existente entre modelos estatísticos que são utilizadas em processamento de sinais e reconhecimento de padrões \cite{Basseville1989349}.   

%Como medidas de similaridade avaliamos métricas clássicas {\color{red} CITAR TRABALHOS IMPORTANTES E REVISTAS ISI}, como a norma $L_2$, bem como medidas de divergência estocástica{\color{red} CITAR FONTES ATUAIS SE TIVER}  {\color{red}}. Essas últimas medem o grau de similaridade existente entre distribuições de probabilidade. Embora sejam empregadas em análise estatística de dados, processamento de sinais e teoria da informação, as medidas de divergência têm sido pouco exploradas na avaliação de similaridade entre formas, em particular no contexto de recuperação de imagens pelo conteúdo.  


%Uma área de pesquisa particularmente relevante em visão computacional é a que trata do desenvolvimento de algoritmos capazes de avaliar o grau de similaridade existente entre as formas de maneira análoga àquela realizada pelos humanos. Tal empreendimento envolve representar, ou descrever, computacionalmente o conteúdo dessas imagens e obter medidas que possibilitem avaliar a similaridade entre as mesmas com base em suas representações. Esses dois aspectos são o foco desse trabalho de pesquisa.



O objetivo geral deste trabalho é investigar e desenvolver métodos de extração de características multiescala e de avaliação de similaridade de formas considerando sua aplicabilidade em experimentos de agrupamento, classificação e recuperação (\emph{CBIR}). Quanto aos objetivos específicos destacamos:

\begin{itemize}
\item Desenvolver descritores multiescala de formas a partir de características do contorno que envolvam informação sobre a curvatura, energia de dobramento e entropia diferencial;

\item Empregar técnicas de visualização de dados e medidas de avaliação da qualidade de agrupamentos para quantificar a capacidade discriminativa dos descritores estudados; 


\item Propor uma nova função de custo mínimo que auxilia os algoritmos de otimização na busca de soluções ótimas para os conjuntos de parâmetros multiescala de descritores formas;

\item Desenvolver de uma metodologia de ajuste automático de parâmetros dos descritores multiescala baseada em otimização evolucionária;

\item Realizar experimentos de classificação e recuperação em uma base pública de imagens de folhas de plantas;

\item Investigar a aplicabilidade de medidas de divergência na avaliação de similaridade de formas em experimentos \emph{CBIR}.
\end{itemize}


\section{Contribuições \label{sec:contrib}}
%1.2 CONTRIBUIÇÕES( LISTA DO QUE VC ALCANÇOU COM O TRABALHO)

Como principais contribuições deste trabalho destacamos:

\begin{itemize}
\item Proposta da entropia multiescala como um novo descritor para análise de formas;

\item Proposta de uma metodologia de ajuste automático de parâmetros de descritores multiescala de formas;

\item Proposta de uma nova função objetivo baseada na qualidade de agrupamentos para guiar  algoritmos de otimização na busca de soluções ótimas dos parâmetros multiescala de descritores formas; 

\item Aplicação das medidas de divergência na análise de similaridade de formas em experimentos \emph{CBIR};

\item Avaliação qualitativa de descritores de formas por ferramentas de análise exploratória visual de agrupamentos;

\item Avaliação qualitativa e quantitativa da capacidade discriminativa de descritores multiescala de formas;

\end{itemize}



%%%%%%%%%%%%%%%%%%%%%%%%%%%%%%%%%%%%%%%%%%%%%%%%%%%%%%%%%%%%%%%%%%%%%%%%%%%%%%%%%%%%%
% Da qualificação
%%%%%%%%%%%%%%%%%%%%%%%%%%%%%%%%%%%%%%%%%%%%%%%%%%%%%%%%%%%%%%%%%%%%%%%%%%%%%%%%%%%%%
\section {Produção Científica \label{sec:prod_cientif}}
\begin{itemize}

\item USHIZIMA, D.; CARNEIRO, A.; SOUZA, M.; MEDEIROS, F. Investigating pill recognition methods for a new national library of
medicine image dataset. In: Advances in Visual Computing. Springer Science Business Media, 2015. p 410-419. Disponível em: <http://dx.doi.org/10.1007/978-3-319-27863-6\_38>

\item SOUZA, M. M. de; MEDEIROS, F. N.; RAMALHO, G. L.; DE PAULA, I. C.; Oliveira, I. N. Evolutionary optimization of a multiscale descriptor for leaf
shape analysis. Expert Systems with Applications, p. –, 2016. ISSN 0957-4174. Disponível
em: <http://www.sciencedirect.com/science/article/pii/S095741741630361X>.

\end{itemize}
\section{Organização da Tese \label{sec:org}}

Os capítulos desta tese seguem a seguinte estrutura:

\noindent \textbf{Capítulo \ref{chap:FUNDA}}:  descrição de conceitos teóricos e ferramentas utilizadas no desenvolvimento, avaliação e testes dos descritores, assim como no desenvolvimento da metodologia proposta para análise de formas.

\noindent \textbf{Capítulo \ref{chap:TINFO}}:  introdução e aplicação de conceitos de teoria da informação como entropia e medidas de divergências utilizadas no desenvolvimento de um novo descritor multiescala para representação e análise de formas.

\noindent \textbf{Capítulo \ref{chap:MatMet}}:  apresentação da metodologia para ajuste automático de parâmetros multiescala de formas por algoritmos de otimização evolucionária. Apresentamos ainda a função objetivo a ser minimizada na busca de soluções ótimas do problema.

