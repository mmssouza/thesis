% !TeX root = ../main.tex
\chapter{Conclusões e Trabalhos Futuros \label{chap:ch5}}

Esta tese abordou a caracterização e análise de formas por meio de descritores multiescala baseados em conceitos de curvatura como a energia de dobramento. Estes descritores foram calculados a partir do contorno de formas binárias de bases de uso geral e uma base pública de folhas. Considerando que estes descritores podem ser utilizados em diferentes aplicações de visão computacional e portanto, distintas bases de imagens, introduzimos uma metodologia versátil de ajuste automático de parâmetros por otimização evolucionária. Vale destacar a importância e ao mesmo tempo a dificuldade inerente ao processo de ajuste manual ou empírico de parâmetros multiescala a um determinado problema ou aplicação. Assim sendo, neste trabalho apresentamos uma alternativa automática para o delineamento do descritor e seus parâmetros às particularidades do problema  e da base em estudo.  
Na metodologia proposta para análise de formas de folhas de plantas, a função objetivo, a ser minimizada no processo de busca dos parâmetros, é fundamentada na medida de qualidade de agrupamento \textit{silhouette}. Como resultado da minimização desta função objetivo, encontramos em um conjunto otimizado de parâmetros de escala dos descritores de forma que são utilizados na análise de formas, em particular de folhas de plantas.  A versatilidade desta metodologia se deve portanto, ao fato de que a mesma pode ser ajustada à função  objetivo ou função custo e moldada assim para as bases de imagens do problema abordado.
O desempenho da metodologia de otimização se mostrou satisfatório e promissor uma vez que os parâmetros otimizados incorporaram características e detalhes sutis das formas, o que foi confirmado pelas técnicas de avaliação qualitativa e quantitativa. De fato o conjunto otimizado de parâmetros, moldados pela minimização da função objetivo,  embute essas nuances da forma. A comprovação desde achado se deu pela considerável melhoria alcançada na organização dos agrupamentos, quantificada pela medida da qualidade de agrupamento e pela elevação na taxa de acerto da classificação das formas ao utilizarmos os descritores otimizados em bases com elevada similaridade de formas entre classes. As versões otimizadas dos descritores estudados discriminaram diferenças de formas dentro de uma mesma classe e entre classes.  
Observamos ainda que determinadas classes de formas apresentaram-se mais desafiadoras que as demais para a representação das mesmas, e isso foi comprovado pela medida \textit{silhouette}  e o arranjo espacial exibido pelas técnicas de visualização exploratória de agrupamentos.
Experimentos com uma base pública de imagem de folhas indicaram a adequação das metodologias propostas em problemas de taxonomia de folhas de plantas. Vale ressaltar que a base de imagens de folhas de plantas Flavia é bastante desafiadora pois a mesma apresenta uma elevada similaridade entre formas de classes distintas. Isso significa que formas de classes distintas não apresentam significativas variações nos contornos das mesmas. Entretanto, essas particularidades podem ser captadas pelos descritores multiescala, os quais se mostraram bastante efetivos no agrupamento e classificação de formas de folhas.
Esta importante característica da base Flavia é portanto um desafio para a metodologia e  para os descritores, de modo geral. 
Ademais, a metodologia proposto constitui uma ferramenta adicional e fonte de informação para taxonomistas discriminarem e agruparem espécies de plantas. 


\subsection{Trabalhos futuros}
A partir deste trabalho, observamos que se desdobram outras ações futuras relacionadas a seguir: 
\begin{itemize}

\item melhoria do processo de otimização dos parâmetros, buscando outras alternativas para a função objetivo, de modo que se reduza o custo computacional da metodologia e melhore as taxas de classificação e recuperação;

\item análise de desempenho da metodologia proposta em outros problemas de visão computacional como reconhecimento automático de células saudáveis e com carcinoma a partir de descritores multiescala de formas;

\item aplicação da metodologia de otimização em classificação taxonômica de outras espécies vegetais e animais;

\item investigação e ajuste dos parâmetros dos algoritmos de otimização de modo que resultem em maior estabilidade na convergência, assim como em maior velocidade de processamento;

\item seleção e testes de outros algoritmos de otimização evolucionária em análise  de formas;

\end{itemize}

\begin{comment}
\textcolor{red}{
In this paper, we introduce a promising methodology for a multiscale descriptor evaluation and used it to investigate the suitability of \emph{NMBE} and \emph{MFD} shape descriptors for CBIR applications. Thus, 
this methodology can be used to evaluate and select shape descriptors for a specific CBIR application.

Our experiments showed that the proposed descriptor can reliably retrieve shapes that exhibit multiscale characteristics similar to those of a given query image. Moreover, the \emph{NMBE} was more robust in distinguishing subtle shape differences within classes than \emph{MFD}.

The relationship between these descriptors and the classes of binary shapes that they represent in a multidimensional space was inferred using the U-matrix. Our findings established that this high-dimensional projection tool provides a qualitative evaluation and visual explanation of how the different classes of shapes are spatially grouped and scattered. Regarding the methodology for quantitative evaluation,  we concluded that the \emph{Silhouette} measure was suitable and fair in numerically assessing the performance of multiscale descriptors and thus inferring  whether a descriptor was able to represent a shape in CBIR applications. 
The qualitative and quantitative evaluation approaches provide valuable information to help elucidate how descriptors represent shapes in a multidimensional space, and therefore we recommend these methods for CBIR experiments.


%\color{red}
We have also investigated  a theoretical relationship between these descriptors and the classes of objects in two public databases of binary images within a multidimensional space by using the U-matrix. Our findings have shown that the high dimensional projection tool, i.e., U-Matrix (SOM map) provides a qualitative evaluation and visual explanation of how the different classes of shapes are spatially grouped or scattered. Regarding the methodology for quantitative evaluation,  we concluded that the \emph{Silhouette} measure was suitable to numerically assess the performance of the multiscale descriptors and thus infer  whether a descriptor is able or not to represent a shape in CBIR applications. 
In fact, the qualitative and quantitative evaluation approaches provided valuable information to help understand how descriptors represent shapes in a multidimensional space.
%\color{black}


The \emph{NMBE} descriptor arises from the bending energy calculated for a shape contour submitted to several stages of smoothing.  Highly discriminating features contribute towards obtaining high clustering or classification accuracy.\color{black} 
This results in a shape signature that allows the use of classical similarity measures at low computational cost.
Since that only the euclidean $L_2$ norm was used as distance metric in our evaluation proposal, we expect to adapt it to other metrics in a future work. 

The CBIR algorithms that were developed can reliably retrieve the candidate image patches exhibiting intensity and morphological characteristics that are most similar to a given query image. The methods described in this paper are able to reliably discriminate among subtle staining differences and spatial pattern distributions. By integrating a newly developed dual-similarity relevance feedback module into the CBIR framework, the CBIR results were improved substantially. By aggregating the computational power of high performance computing (HPC) and cloud resources, we demonstrated that the method can be successfully executed in minutes on the Cloud compared to weeks using standard computers. 

to improve the shape retrieval performance and, furthermore, Figure \ref{fig:descritores}b the weak discriminating property of \emph{MFD}
are more spaced than the ones observed in Figure \ref{fig:descritores}b. Accordingly, the corresponding shapes appears in the U-matrix more scattered for \emph{MFD} than for \emph{NMBE}. Thus, it illustrates how \emph{NMBE} tends to improve the retrieval performance and, furthermore, the weak discriminating property of \emph{MFD}.


In this paper, we used the Euclidean $L_2$ norm, but future research will investigate other distance metrics. 

In this work we propose a method to evaluate similarity between binary shapes by employing divergence measures and combination of shape contour signatures. Finally, our tests led us to conclude that:  a) divergence measures are promising tools for shape similarity evaluation in shape retrieval; b) divergences that present high sensitivity to small variations between similar mass distributions tend to perform worst in shape similarity evaluation than divergences that feature low sensitivity and c) combining different shape signatures improves the shape retrieval Precision in experiments. 

This paper presents an optimization methodology for parameter adjustment of shape descriptors applied to leaf characterization and clustering. Our optimization methodology is a versatile tool for shape analysis that mainly relies on an objective function that can be adapted to different application problems or databases. Here, the minimization of the objective function accomplishes the best parameter set of a given shape descriptor to improve leaf characterization quality. 

The performance evaluation of the optimization methodology led us to conclude that the optimized parameters were able to reveal subtle leaf shape features and overall they have improved cluster organization and classification of plant leaves.
Actually, the optimized descriptors have reliably characterized shapes that exhibited multiscale features and they have also discriminated shape differences within and among leaf classes.
Likewise, the optimized shape descriptors provided a global and robust description on the challenging Flavia data set,  despite it presents a high between class similarity.

A relevant visual analysis of the proposed methodology linked the optimized and non-optimized NMBE with the classes of binary shapes that they represented in a multidimensional space by using U-matrices. This high-dimensional projection tool provided a qualitative evaluation and visual explanation on how the different classes of shapes were better spatially grouped and scattered due to the optimization methodology. Regarding the quantitative performance evaluation,  we have also observed that the \emph{silhouette} measure was suitable to numerically assess the multiscale shape descriptor and thus infer whether it was able to characterize leaf shapes or not. Our findings indicated that the optimized NMBE and IDSC were suitable to characterize leaf shapes and furthermore they may provide shape signatures to address leaf  taxonomy problems. Moreover, the proposed methodology is an additional tool and source of information for plant taxonomist to discriminate and group leaf species.
}
\end{comment}