% !TeX root = ../main.tex
\chapter{Conclusões e Trabalhos Futuros \label{chap:ch5}}

Esta tese abordou a caracterização e análise de formas por meio de descritores multiescala baseados em conceitos de curvatura, energia de dobramento e entropia os quais foram calculados a partir do contorno de formas binárias. Considerando que estes descritores podem ser utilizados em diferentes aplicações e portanto, distintas bases de imagens, introduzimos uma metodologia versátil de ajuste automático de parâmetros por otimização evolucionária.
A proposta da função objetivo a ser minimizada no processo de busca dos parâmetros é fundamentada na medida de qualidade de agrupamento \textit{silhouette}. Vale ressaltar que minimização desta função objetivo resulta no conjunto de parâmetros de escala ótimos dos descritores de forma que serão utilizados no problema em análise. A versatilidade desta metodologia se deve portanto, ao fato de que a mesma pode ser ajustada à  função  objetivo ou função custo e moldada assim para as bases de imagens do problema abordado.
A avaliação de desempenho da metodologia de otimização mostrou que os parâmetros otimizados revelaram características e detalhes sutis das formas. De fato o conjunto otimizado de parâmetros, moldados pela minimização da função objetivo,  embute essas nuances da forma. A comprovação desde achado se deu pela considerável melhoria alcançada na organização dos agrupamentos e classificação das formas ao utilizarmos os descritores otimizados em bases com elevada similaridade de formas entre classes. As versões otimizadas dos descritores estudados discriminaram diferenças de formas dentro de uma mesma classe e entre classes.  
Observamos ainda que determinadas classes de formas apresentaram-se mais desafiadoras que as demais para a representação das mesmas, e isso foi comprovado pela medida \textit{silhouette}  e o arranjo espacial exibido pelas técnicas de visualização exploratória de agrupamentos.
Experimentos com uma base pública de imagem de folhas indicaram a adequação das metodologias propostas em problemas de taxonomia de folhas de plantas.
Ademais, estas metodologias constituem uma ferramenta adicional e fonte de informação para taxonomistas discriminarem e agruparem espécies de plantas. 


\subsection{Trabalhos futuros}
A partir deste trabalho, observamos que se desdobram outras ações futuras relacionadas a seguir: 
\begin{itemize}

\item melhoria do processo de otimização dos parâmetros, buscando outras alternativas para a função objetivo, de modo que se reduza o custo computacional do mesmo e melhore as taxas de classificação e recuperação;

\item investigação e ajuste dos parâmetros dos algoritmos de otimização de modo que resultem em maior estabilidade na convergência, assim como em maior velocidade de processamento;

\end{itemize}


%Um dos desafios no campo de visão computacional é fazer com que computadores avaliem o grau de similaridade existente entre formas extraídas de imagens. Embora não esteja completamente elucidado como os sistemas de visão biológicos realizam tal tarefa, busca-se para esta finalidade obter medidas de similaridade que operem de forma fidedigna aos sistemas de visão biológicos. 

