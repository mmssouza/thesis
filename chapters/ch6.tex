
\chapter{Estudos de caso}
Este capítulo apresenta dois estudos de caso em que a aplicação da metodologia proposta é ilustrada. Iniciamos o capítulo com o primeiro estudo de caso: classificação de espécies através de atributos multiescala das folhas. Em seguida, apresentados o segundo estudo de caso: o reconhecimento de medicamentos a partir de atributos de forma pílulas.

\section{Classificação de espécies vegetais}

\subsection{Introdução}

A identificação e classificação de espécies de plantas é um problema de interesse da biologia. Estima-se que existam certa de $298.000$ espécies vegetais terrestres, embora estejam catalogadas um total de $215.000$ (MORA et al.,2011). Há, portanto, temas de pesquisa em aberto e trabalho por fazer, particularmente, nos países com ecossistemas tropicais aonde há grande biodiversidade vegetal.

A abordagem para a classificação de espécies de plantas é a taxonomia, ou seja, o treinamento de especialistas, denominados taxonomistas, em identificar as espécies manualmente  através das características observadas nas plantas. A deficiência desse método é o aspecto subjetivo do taxonomista, que é o especialista que julga as diferenças entre espécies (GOMES et al.,2013). Ademais, normalmente os taxonomistas são especialistas em identificar um pequeno grupo de espécies, sendo assim impossível que os mesmos classifiquem entre diversas espécies \cite{Cope20127562}.

Historicamente, as amostras de espécies foram sempre armazenadas em arquivos físicos, denominados de herbários, para posterior classificação, tornando o acesso às amostras difícil e lento. Com a criação das bases de dados digitais, a informação tornou-se facilmente acessível aos pesquisadores em qualquer parte do mundo, o que têm permitido o desenvolvimento de métodos computacionais automatizados de classificação de espécies baseados em reconhecimento de padrões através de características morfológicas.

Da necessidade de se aplicar técnicas de reconhecimente de padrões na classificação de espécies vegetais, surgiu a idéia de se extrair características do contorno da forma de folhas, provenientes de uma base de imagens real coletadas em campo, através das técnicas multiescala.

\subsection{Base de imagens}

