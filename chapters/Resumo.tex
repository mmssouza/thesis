% !TeX root = ../main.tex

%Dentre os atributos da imagem de um objeto a forma desempenha um papel crucial na percepção visual humana. Esse atributo tem sido amplamente explorado em visão computacional em aplicações como classificação, reconhecimento e recuperação pelo conteúdo de objetos. Um tópico de particular interesse em análise computacional de formas é o desenvolvimento de algoritmos capazes de avaliar o grau de similaridade existente entre as mesmas de maneira análoga àquela realizada por um ser humano. Nesta tese exploramos técnicas da teoria da informação para este fim. Propomos um novo descritor baseado no contorno das formas empregando o conceito de entropia diferencial da curvatura multiescala. Aplicamos também medidas de divergência na avaliação da similaridade entre formas com base em histogramas de assinaturas extraídas dos seus contornos. Avaliamos nossas propostas em bases de imagens públicas através de técnicas de visualização de dados, de medidas de avaliação de agrupamentos e do desempenho em experimentos de recuperação de formas pelo conteúdo. Concluímos que conceitos da teoria da informação podem ser aplicados com sucesso na descrição e na avaliação de similaridade de formas.

\begin{comment}
A popularização dos dispositivos portáteis de aquisição de imagens tem contribuído para um grande volume de informação multimídia disponível na internet. Dada a limitação dos tradicionais motores de busca em organizar e acessar tal volume de informação, os sistemas de recuperação de imagens através do conteúdo (\emph{CBIR}) despontam como uma proposta promissora para essa finalidade. Esses sistemas realizam buscas por imagens a partir do conteúdo visual para recuperar aquelas que sejam de interesse do usuário. Dois tópicos são de grande relevância em \emph{CBIR}: os métodos de extração de características das imagens e as medidas para avaliação de similaridade. Divergentes estocásticos são funcionais que medem a similaridade entre distribuições de probabilidades. Embora frequentemente aplicados em estatística, teoria da informação e processamento de sinais, a aplicação de divergentes em \emph{CBIR} tem sido pouco explorado. Este trabalho se propõe a investigar a aplicação de tais medidas na avaliação de similaridade de imagens através de características extraídas do contorno da forma. Tendo como descritores os histogramas de assinaturas do contorno das formas, foram conduzidos experimentos de recuperação de imagens pelo conteúdo em bases de imagens binárias empregando-se tais medidas. Os resultados dessa investigação demonstram que a avaliação de similaridade de formas em experimentos \emph{CBIR} através de divergentes é viável.         
\end{comment}

%%%%%%%%%%%%%%%%%%%%%%%%%%%%%%%%%%%%%%%%%%%%%%%%%%%%%%%%%%%%%%%%%%%%%%%%%%%%%%%%%%%%%%%%%%%%%%%%%%%%%%%%%%%
% Da qualificação
%%%%%%%%%%%%%%%%%%%%%%%%%%%%%%%%%%%%%%%%%%%%%%%%%%%%%%%%%%%%%%%%%%%%%%%%%%%%%%%%%%%%%%%%%%%%%%%%%%%%%%%%%%%
\begin{comment}
A forma é um importante atributo do sistema visual de primatas para o reconhecimento de objetos. Este atributo tem sido amplamente explorado em  aplicações de visão computacional  tais como classificação, reconhecimento e recuperação de imagens pelo conteúdo. Um sistema de visão computacional para o reconhecimento de objetos realiza a análise da forma que consiste  na descrição ou representação da mesma e na análise de similaridade entre formas com base em suas representações. 
Nesta trabalho investigamos o uso de conceitos da teoria da informação em análise de formas e propomos uma nova 
metodologia para análise de formas baseada na energia de dobramento multiescala e conceitos de entropia. Aplicamos também medidas de divergência para avaliar similaridade entre as formas com base em histogramas de assinaturas extraídas dos seus contornos.    Além disso, introduzimos uma metodologia de otimização evolucionária para ajuste automático de parâmetros de descritores multiescala. Realizamos experimentos em formas de bases públicas de imagens. A  avaliação de desempenho dos algoritmos e resultados consistiu de técnicas de visualização de dados, medidas de avaliação de agrupamentos, classificação e recuperação de formas baseada conteúdo. A análise visual exploratória dos agrupamentos mostrou que a metodologia de otimização utilizada no ajuste automático dos parâmetros dos descritores melhorou os resultados de agrupamento e recuperação de formas. Experimentos de classificação supervisionada e não supervisionada alcançara elevadas taxas de Precisão e Revocação, assim como da medida Bulls-eye com o usos de descritores otimizados. Finalmente, os resultados nos levaram a concluir que conceitos da teoria da informação são adequados em aplicações de análise de formas e análise de similaridade das mesmas.
\end{comment}

\begin{comment}
A forma é um importante atributo do sistema visual de primatas para o reconhecimento de objetos. Este atributo tem sido amplamente explorado em  aplicações de visão computacional, tais como classificação, reconhecimento e recuperação de imagens pelo conteúdo. Um sistema de visão computacional para o reconhecimento de objetos realiza a análise da forma, que consiste na descrição ou representação da mesma e na análise de similaridade entre formas.  Este trabalho introduz um método automático para o ajuste de descritores multiescala. Utilizando algoritmos de otimização evolucionários, o método proposto busca pelo ajuste dos parâmetros de escala do descritor energia de dobramento multiescala que propicia a melhor configuração possível de agrupamento das formas. Realizamos experimentos em formas de bases públicas de imagens com o descritor em questão. Na avaliação de desempenho dos algoritmos utilizamos técnicas de visualização de dados, medidas de avaliação de agrupamentos e experimentos de classificação e recuperação de formas pelo conteúdo. A análise visual exploratória dos agrupamentos mostrou que a metodologia de otimização melhorou os resultados de agrupamento e recuperação de formas. Experimentos de classificação supervisionada e não supervisionada com os descritores otimizados alcançaram elevadas taxas de precisão e revocação, assim como da medida Bulls-eye. Tendo a função objetivo composta por métricas de qualidade de agrupamentos, a otimização de descritores de forma melhora a representação das formas nos aspectos de coesão intraclasse e separação entre classes, o que reflete positivamente no desempenho em experimentos de classificação e recuperação de formas.
\end{comment}

A forma é um importante atributo do sistema visual de primatas que tem sido amplamente explorado em visão computacional em aplicações de classificação, reconhecimento e recuperação de imagens pelo conteúdo. Um sistema de visão computacional para o reconhecimento de objetos realiza 
a análise da forma, que consiste na descrição ou representação da mesma e na análise 
de similaridade entre formas. Um aspecto importante em análise de forma é a adequação
do descritor ao problema de reconhecimento de padrões de interesse, porém há uma carência de métodos que sistematizem essa adequação. Este trabalho propõe um método automático para o ajuste dos parâmetros
de descritores de formas com otimização evolutiva. A aplicabilidade do referido método é investigada na adequação de um descritor multiescala de 
forma ao problema de identificação das espécies de plantas a partir das suas folhas, sendo seu desempenho avaliado por técnicas de visualização de dados, 
medidas de avaliação de agrupamentos e experimentos de classificação e recuperação de formas pelo conteúdo. A análise visual exploratória dos agrupamentos mostrou 
que a metodologia de otimização melhora os resultados de agrupamento e recuperação de formas. Já os experimentos de classificação com os descritores otimizados alcançaram elevadas
taxas de precisão e revocação, assim como da medida Bulls-eye. Tendo como função objetivo métricas de qualidade de
agrupamentos, a otimização de descritores de forma melhora a representação das formas nos aspectos de coesão intraclasse e separação entre classes, o que reflete positivamente no desempenho em experimentos de classificação e recuperação de formas.
