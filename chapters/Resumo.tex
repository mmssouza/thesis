% !TeX root = ../main.tex

%Dentre os atributos da imagem de um objeto a forma desempenha um papel crucial na percepção visual humana. Esse atributo tem sido amplamente explorado em visão computacional em aplicações como classificação, reconhecimento e recuperação pelo conteúdo de objetos. Um tópico de particular interesse em análise computacional de formas é o desenvolvimento de algoritmos capazes de avaliar o grau de similaridade existente entre as mesmas de maneira análoga àquela realizada por um ser humano. Nesta tese exploramos técnicas da teoria da informação para este fim. Propomos um novo descritor baseado no contorno das formas empregando o conceito de entropia diferencial da curvatura multiescala. Aplicamos também medidas de divergência na avaliação da similaridade entre formas com base em histogramas de assinaturas extraídas dos seus contornos. Avaliamos nossas propostas em bases de imagens públicas através de técnicas de visualização de dados, de medidas de avaliação de agrupamentos e do desempenho em experimentos de recuperação de formas pelo conteúdo. Concluímos que conceitos da teoria da informação podem ser aplicados com sucesso na descrição e na avaliação de similaridade de formas.

\begin{comment}
A popularização dos dispositivos portáteis de aquisição de imagens tem contribuído para um grande volume de informação multimídia disponível na internet. Dada a limitação dos tradicionais motores de busca em organizar e acessar tal volume de informação, os sistemas de recuperação de imagens através do conteúdo (\emph{CBIR}) despontam como uma proposta promissora para essa finalidade. Esses sistemas realizam buscas por imagens a partir do conteúdo visual para recuperar aquelas que sejam de interesse do usuário. Dois tópicos são de grande relevância em \emph{CBIR}: os métodos de extração de características das imagens e as medidas para avaliação de similaridade. Divergentes estocásticos são funcionais que medem a similaridade entre distribuições de probabilidades. Embora frequentemente aplicados em estatística, teoria da informação e processamento de sinais, a aplicação de divergentes em \emph{CBIR} tem sido pouco explorado. Este trabalho se propõe a investigar a aplicação de tais medidas na avaliação de similaridade de imagens através de características extraídas do contorno da forma. Tendo como descritores os histogramas de assinaturas do contorno das formas, foram conduzidos experimentos de recuperação de imagens pelo conteúdo em bases de imagens binárias empregando-se tais medidas. Os resultados dessa investigação demonstram que a avaliação de similaridade de formas em experimentos \emph{CBIR} através de divergentes é viável.         
\end{comment}

%%%%%%%%%%%%%%%%%%%%%%%%%%%%%%%%%%%%%%%%%%%%%%%%%%%%%%%%%%%%%%%%%%%%%%%%%%%%%%%%%%%%%%%%%%%%%%%%%%%%%%%%%%%
% Da qualificação
%%%%%%%%%%%%%%%%%%%%%%%%%%%%%%%%%%%%%%%%%%%%%%%%%%%%%%%%%%%%%%%%%%%%%%%%%%%%%%%%%%%%%%%%%%%%%%%%%%%%%%%%%%%
{\color{red} POR AQUI A TRADUÇÃO FIEL DO ABSTRACT MARCELO}
A forma é um atributo importante no reconhecimento visual de objetos pelos primatas que tem sido amplamente explorada em  aplicações de visão computacional  como classificação, reconhecimento e recuperação de imagens pelo conteúdo. Um sistema de visão computacional para o reconhecimento de formas deve ser capaz de analisá-las, o que envolve representá-las ou descrevê-las computacionalmente, além de medir o grau de similaridade entre as mesmas com base em suas representações. Nesta trabalho exploramos técnicas da teoria da informação em análise de formas. Propomos um novo descritor baseado na análise multiescala da curvatura do contorno das formas e no conceito de entropia diferencial. Aplicamos também medidas de divergência na avaliação da similaridade entre as formas com base em histogramas de assinaturas extraídas dos seus contornos. As metodologias propostas, para análise e reconhecimento de formas, foram testadas em bases de imagens públicas  e validadas por meio de técnicas de visualização de dados, medidas de avaliação de agrupamentos, de classificação e de recuperação de formas baseada conteúdo. As avaliações qualitativa e quantitativa dos resultados realizados mostraram a viabilidade da aplicação de conceitos da teoria da informação em análise de formas, em particular na descrição e análise de similaridade das mesmas.
