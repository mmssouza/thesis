A popularização dos dispositivos portáteis de aquisição de imagens tem contribuído para um grande volume de informação multimídia disponível na internet. Dada a limitação dos tradicionais motores de busca em organizar e acessar tal volume de informação, os sistemas de recuperação de imagens através do conteúdo (\emph{CBIR}) despontam como uma proposta promissora para essa finalidade. Esses sistemas realizam buscas por imagens a partir do conteúdo visual para recuperar aquelas que sejam de interesse do usuário. Dois tópicos são de grande relevância em \emph{CBIR}: os métodos de extração de características das imagens e as medidas para avaliação de similaridade. Divergentes estocásticos são funcionais que medem a similaridade entre distribuições de probabilidades. Embora frequentemente aplicados em estatística, teoria da informação e processamento de sinais, a aplicação de divergentes em \emph{CBIR} tem sido pouco explorado. Este trabalho se propõe a investigar a aplicação de tais medidas na avaliação de similaridade de imagens através de características extraídas do contorno da forma. Tendo como descritores os histogramas de assinaturas do contorno das formas, foram conduzidos experimentos de recuperação de imagens pelo conteúdo em bases de imagens binárias empregando-se tais medidas. Os resultados dessa investigação demonstram que a avaliação de similaridade de formas em experimentos \emph{CBIR} através de divergentes é viável.         

% Dentre as características das imagens, a forma é de grande relevância em diversas aplicações de visão computacional. Há diversos métodos propostos na literatura para extração de características das formas, mas apenas recentemente a combinação dos métodos, para compor descritores, têm sido investigado.