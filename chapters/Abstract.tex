% !TeX root = ../main.tex

 Shape is an important of the primate visual system visual attribute for object recognition. This attribute has been widely explored in computer vision applications, such as object classification, recognition and content-based image retrieval. A computer vision system for object recognition performs shape analysis, which encompasses shape description or representation, and similarity evaluation shapes similarity. In this work, we explore useful information theory concepts in shape analysis. Thus, we propose a new methodology for shape analysis which relies on the multiscale bending energy and the concept of differential entropy. In addition, we introduce an evolutionary optimization methodology for parameter adjustment of multiscale shape descriptors. We also apply divergence measures to evaluate shapes similarity based on histograms of its contour signatures.
Experiments were conducted on public benchmark data sets with the normalized multiscale bending energy and the multiscale differential entropy descriptors.  The  performance assessment was attained with data visualization techniques, clustering quality evaluation measures and shape classification and retrieval experiments.
The visual exploratory data analysis techniques showed that the proposed methodology for parameter adjustment of shape descriptors  improved shape clustering and retrieval. Moreover, unsupervised classification experiments accomplished high Precision and Recall rates as well as Bulls-eye scores with the optimized parameters.  
 We carried out experiments on shapes from public image data sets  and the performance assessment assessed through data visualization techniques, clustering quality evaluation measures and shape classification and retrieval experiments. The results led us to conclude that information theory concepts are suitable for shape and similarity analysis. 

