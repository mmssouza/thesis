% !TeX root = ../main.tex
\begin{comment}
 Shape is an important atribute of the primate visual system for object recognition. This atribute has been widely explored in computer vision applications, such as object classification, recognition and content-based image retrieval. A computer vision system for object recognition performs shape analysis, which encompasses shape description or representation and shape similarity analysis. In this work, we investigate the use of information theory concepts in shape analysis and thus, we propose a new methodology for shape analysis which relies on the multiscale bending energy and the concept of differential entropy. We also apply divergence measures to evaluate shapes similarity based on histograms of its contour signatures.
Moreover, we introduced an evolutionary optimization methodology for parameter adjustment of multiscale shape descriptors. 
We carried out experiments on shapes from public image data sets. The methodology for performance assessment of the algorithms and results comprised data visualization techniques, clustering quality evaluation measures and shape classification and retrieval experiments. 
The visual exploratory data analysis techniques showed that the proposed methodology for parameter adjustment of shape descriptors  improved shape clustering and retrieval. Moreover, unsupervised classification experiments accomplished high Precision and Recall rates as well as Bulls-eye scores with the optimized parameters.  Finally, the results led us to conclude that information theory concepts are suitable for shape and similarity analysis. 
\end{comment}


Shape is an important attribute of the primate visual system that has been widely explored in computer vision applications, such as object classification, recognition and content-based image retrieval. A computer vision system for object recognition performs shape analysis, which encompasses shape description or representation and shape similarity analysis. A relevant aspect in shape analysis is to adjust the descriptor to the pattern recognition problem of interest, even though there is a lack of consistent methods for doing so.  This work introduces an automatic method to setup multiscale shape descriptor through evolutionary optimization. The method was applied to adjust a multiscale shape descriptor to the problem of leaf-based plant specimen identification, where data visualization techniques, clustering quality metrics and shape classification and retrieval experiments were used to assess its performance. The visual exploratory data analysis techniques showed that the proposed methodology improved shape clustering and retrieval. Moreover, supervised and unsupervised classification experiments accomplished high precision and recall rates as well as Bulls-eye scores with the optimized parameters. Thus, taking as objective function clustering quality metrics, shape descriptors optimization leads to improvement in shapes representation in terms of intraclass coesion and inter-class separation, which reflects positively in shape classification experiments performance.    

