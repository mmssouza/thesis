% !TeX root = ../documento.tex

Shape is an important attribute of the primate visual system that has been widely explored in computer vision applications, such as object classification, recognition and content-based image retrieval. A computer vision system for object recognition performs shape analysis, which encompasses shape description or representation and shape similarity analysis. A relevant aspect in shape analysis is to adjust the descriptor to the pattern recognition problem of interest, even though there is a lack of consistent methods for doing so.  This work introduces an automatic method to setup multiscale shape descriptor through evolutionary optimization. The method was applied to adjust a multiscale shape descriptor to the problem of leaf-based plant specimen identification, where data visualization techniques, clustering quality metrics and shape classification and retrieval experiments were used to assess its performance. The visual exploratory data analysis techniques showed that the proposed methodology improved shape clustering and retrieval. Moreover, supervised and unsupervised classification experiments accomplished high precision and recall rates as well as Bulls-eye scores with the optimized parameters. Thus, taking as objective function clustering quality metrics, shape descriptors optimization leads to improvement in shapes representation in terms of intraclass coesion and inter-class separation, which reflects positively in shape classification experiments performance.    

\keywords{Computational vision. Pattern recognition. Shape analysis.  Multiscale descriptors. Evolutionary optimization.  
}