% !TeX root = ../documento.tex

A forma é um importante atributo do sistema visual dos primatas que tem sido amplamente explorado em visão computacional em aplicações de classificação, reconhecimento e recuperação de imagens pelo conteúdo. Um sistema de visão computacional para o reconhecimento de objetos realiza 
a análise da forma, que consiste na descrição ou representação da mesma e na análise 
de similaridade entre formas. Um aspecto importante em análise de forma é a adequação
do descritor ao problema de reconhecimento de padrões de interesse, porém há uma carência de métodos que sistematizem essa adequação. Este trabalho propõe um método automático para o ajuste dos parâmetros
de descritores de formas com otimização evolutiva. A aplicabilidade do referido método é investigada na adequação de um descritor multiescala de 
forma ao problema de identificação das espécies de plantas a partir das suas folhas, sendo seu desempenho avaliado por técnicas de visualização de dados, 
medidas de avaliação de agrupamentos e experimentos de classificação e recuperação de formas pelo conteúdo. A análise visual exploratória dos agrupamentos mostrou 
que a metodologia de otimização melhora os resultados de agrupamento e recuperação de formas. Já os experimentos de classificação com os descritores otimizados alcançaram elevadas
taxas de precisão e revocação, assim como da medida bulls-eye. Tendo como função objetivo métricas de qualidade de
agrupamentos, a otimização de descritores de forma melhora a representação das formas nos aspectos de coesão intraclasse e separação entre classes, o que reflete positivamente no desempenho em experimentos de classificação e recuperação de formas.

% Separe as palavras-chave por ponto
\palavraschave{ Visão computacional. Reconhecimento de padrões. Análise de formas. Descritores multiescala. Otimização evolucionária.}